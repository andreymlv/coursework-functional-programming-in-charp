\section*{Введение}
Необходимо решить нелинейное уравнение с двумя неизвестными.

В математике решить уравнение - значит найти его решения, которое представляет собой значения (числа, функции, множества и т.д.), удовлетворяющие условию уравнения, состоящим, как правило, из двух выражений, связанных знаком равенства.
При поиске решения одна или несколько переменных обозначаются как неизвестные. 
Решение - это присвоение значений неизвестным переменным, которое делает равенство в уравнении истинным.
Другими словами, решение - это значение или набор значений (по одному для каждого неизвестного), таких, что при замене неизвестных уравнение становится равенством. 

Каждый день человек встречает на своём пути уравнения типа $y=\cos{\tan{\sin{x}}}$ (имеет множество решений), $y=\sqrt{x}$, и было бы прекрасно, если такие задачки решала машина. В настоящее время легко найти области, где решение уравнений было бы чуть-ли не основополагающей задачей. Поэтому можно назвать данную работу 

Цель (решить что-то используя то-то).
Цель данной работы - решить любое нелинейное уравнение используя методы вычислительной математики

Задачи (этапы, которые нужно пройти, чтобы достичь цели):

\begin{itemize}
	\item Изучить литературу по данной теме.
	\item Изучить методы и алгоритмы решения задачи.
	\item Описать сложность для каждого представленного алгоритма в \(O\)-нотации.
\end{itemize}
\addcontentsline{toc}{section}{Введение}

\newpage
\section{Функциональная парадигма программирования}

\subsection{История}

Первый высокоуровневый функциональный язык программирования LISP был разработан в конце 1950-х годов для научных компьютеров серии IBM 700/7000 Джоном Маккарти, когда он учился в Массачусетском технологическом институте (MIT). \cite{HistoryofLISP}

Язык обработки информации (IPL), 1956, иногда упоминается как первый функциональный язык программирования.
Это язык в стиле ассемблера для работы со списками символов. \cite{ModelsofMyLife}
В нём действительно есть понятие генератора, которое сводится к функции, которая принимает функцию в качестве аргумента, и, поскольку это язык уровня ассемблера, то код может быть данными, поэтому IPL можно рассматривать как язык, имеющий функции более высокого порядка.

В 1970-х годах Гай Л. Стил и Джеральд Джей Сассман разработали язык Scheme, и использовали его в своём курсе, основанном на книге 1985 года "Структура и интерпретация компьютерных программ"\cite{SICP}. Scheme был первым диалектом LISP, который использовал лексическую область видимости и требовал оптимизацию хвостовой рекурсией, которые рекомендуются в функциональном программировании. Книга обучает фундаментальными принципами компьютерного программирования, включая рекурсию, абстракцию, модульность, проектирование и реализация языков программирования.

\subsection{Функциональное программирование}

Функциональное программирование - это:

\begin{itemize}
	\item парадигма программирования, характеризующаяся использованием математических функций и избеганием побочных эффектов.
	\item стиль программирования, который использует только чистые функции без побочных эффектов.
\end{itemize}

Побочные эффекты - это всё, что делает функция, кроме возврата значения, например:

\begin{enumerate}
	\item Отправка электронного письма.
	\item Чтение файла.
	\item Выполнение веб-запроса.
\end{enumerate}

Побочные эффекты приводят к тому, что программист не может предсказать, что произойдёт в его программе.

Чистые функции - это функции, которые зависят только от своих аргументов и не имеют никаких побочных эффектов. 
Одни и те же аргументы всегда будут выдавать одно и то же возвращаемое значение. 

Из определения можно подумать, что функциональные языки программирования не имеют грязных функций. 
Но это не так - такой язык бесполезен в реальном мире.

\newpage
\section{Выбор программного обеспечения}

\subsection{Язык программирования}

Так как идея разработки пришла из прочтения книг про функциональное программирование, то и было бы логично применять в данной работе языки, которые являются либо частично функциональными, либо чисто функциональными.

Но одно из требований написания курсовой работы - применять язык программирования C\#, который не является функциональным. В нём есть инструменты, с помощью которых можно писать функциональные методы, но это приходится всё обёртывать в статические классы, которые служат собой обычным именованным пространством.

Так же, стоит отметить сложность разработки на C\# из-за того, что в этом языке система типов является не такой мощной, как в других языках программирования. Например, нельзя создать класс, который будет наследовать все свойства Integer, или Double. В целом, нет интерфейса INumber, с помощью которого можно было бы показать всю мощь объектно-ориентированного языка программирования. Конечно, можно самому создать такой интерфейс, но тогда сложность проекта возросла, из-за интерфейсов, классов, которые вообще никак не относятся к теме разговора курсовой работы.

Как альтернативу C\#, лучше взять какую-нибудь Java, в которой есть хотя бы интерфейс Number.

Можно рассмотреть язык программирования Python, который имеет богатый функционал, более удобные инструменты функционального программирования. 

А если рассматривать не C-подобные языки программирования, то можно было бы использовать прекрасный язык программирования Racket. Маленький минус - он Lisp-подобный, и, следовательно, необходимо опыт, для того, чтобы понимать такой код. В остальном - язык имеет одни плюсы. Богатая документация, удобная среда разработки, поддерживает множество парадигм программирования.

\subsection{Операционная система}

На сегодняшний день почти все языки программирования имеют кроссплатформенную поддержку почти всех популярных операционных систем (BSD, Windows, Linux, MacOS).

Разработка велась на операционной системе Windows 10.

\subsection{Инструменты}

В качестве инструмента написания программ было выбрано несколько программ:

\begin{itemize}
	\item Microsoft Visual Studio
	\item Microsoft Visual Studio Code
	\item JetBrains Rider
\end{itemize}

Microsoft Visual Studio поставляет компилятор и интерпретатор для языка C\# под операционную систему Windows.

Microsoft Visual Studio Code является легковесным редактором текста. В нём есть удобные расширения, с помощью которых я исправлял грамматические ошибки в комментариях к каждому методу.

JetBrains Rider является моим основным средством разработки. Платный, но для студентов предоставляется студенческая бесплатная лицензия. В этой среде разработки имеется удобный дебаггер, статический анализатор кода, который помогает найти ошибки в самом тесте кода. Удобные подсказки и автозаполнение кода увеличивают скорость и качество разработки кода.

В качестве системы контроля версий был выбран Git. Является стандартом в современной разработке софта. Весь текст курсовой работы и сам проект находится в сервисе GitHub\cite{GitHub}.

\newpage
\section{Проектирование}

\newpage
\section{Разработка}

Какие классы, методы.

Зачем они нужны в данной программе.

\newpage
\section{Тестирование}

Протестировать функции на разных данных с разными объёмами.

\newpage
\section{Характеристики ПК}

Все тесты были произведены на ПК с такими характеристиками:

\begin{itemize}
	\item ЦПУ - Intel(R) Core(TM) i5-8265U CPU:
	\begin{enumerate}
		\item Базовая скорость: 1.8 ГГц
		\item Количество ядер: 4
		\item Кэш первого уровня: 256 Кб
		\item Кэш второго уровня: 1 Мб
		\item Кэш третьего уровня: 6 Мб
	\end{enumerate}
	\item ОЗУ:
	\begin{enumerate}
		\item Объём: 16 Гб
		\item Скорость: 2400 МГц
	\end{enumerate}
	\item ЗУ - Samsung SSD 970 EVO Plus
	\begin{enumerate}
		\item Объём: 250 Гб
		\item Среднее время отклика: 1.1 мс
	\end{enumerate}
\end{itemize}

\newpage
\section*{Заключение}
Добились ли мы целей, как можно продолжить изучение данной темы.
\addcontentsline{toc}{section}{Заключение}